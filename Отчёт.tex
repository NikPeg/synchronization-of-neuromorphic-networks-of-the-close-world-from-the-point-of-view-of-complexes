\documentclass[draft]{article}
\usepackage{cmap}
\usepackage[T1,T2A]{fontenc}
\usepackage[utf8]{inputenc}
\usepackage[russian]{babel}
\usepackage[left=2cm,right=2cm,top=2cm,bottom=2cm,bindingoffset=0cm]{geometry}
\usepackage{tikz}


\usepackage{setspace,amsmath}
\usepackage{lipsum}
\usepackage[usestackEOL]{stackengine}
\usepackage{kantlipsum}
\usepackage{graphicx}
\usepackage{caption}
\usepackage{float}
\usetikzlibrary{positioning}
\graphicspath{{pictures/}}
\DeclareGraphicsExtensions{.pdf,.png,.jpg}
\newcommand\zz[1]{\par{\normalsize\strut #1} \hfill\ignorespaces}
\addto\captionsrussian{\def\refname{Список использованных источников}}
\newcommand{\subtitle}[1]{%
  \posttitle{%
    \par\end{center}
    \begin{center}\Large#1\end{center}
   }%
}
\newcommand{\subsubtitle}[1]{%
  \preauthor{%
    \begin{center}
    \large #1 \vskip0.5em
    \begin{tabular}[t]{c}
    }%
}
\begin{document}
 
\begin{center}
\textbf{
ПРАВИТЕЛЬСТВО РОССИЙСКОЙ ФЕДЕРАЦИИ\\
НАЦИОНАЛЬНЫЙ ИССЛЕДОВАТЕЛЬСКИЙ УНИВЕРСИТЕТ\\
«ВЫСШАЯ ШКОЛА ЭКОНОМИКИ»\\
Факультет компьютерных наук
Образовательная программа «Программная инженерия»\\
(ВШЭ ФКН ПИ)}\\
\end{center}
УДК 004.852
\bigskip
\zz{СОГЛАСОВАНО}УТВЕРЖДАЮ
\zz{Руководитель,}Академический руководитель
\zz{Стажер-исследователь,}образовательной программы
\zz{приглашённый лектор}«Программная инженерия»
\zz{\noindent\rule{3cm}{0.4pt} О. Н. Качан}профессор департамента программной
\zz{«\noindent\rule{1cm}{0.4pt}»\noindent\rule{2cm}{0.4pt}20\noindent\rule{0.5cm}{0.4pt}г.}инженерии, канд. техн. наук
\zz{~}\noindent\rule{3cm}{0.4pt} В.В. Шилов
\zz{~}«\noindent\rule{1cm}{0.4pt}»\noindent\rule{2cm}{0.4pt}20\noindent\rule{0.5cm}{0.4pt}г.
\begin{center}
\topskip=0pt
\vspace*{\fill}
\textbf{ОТЧЕТ\\
О НАУЧНО-ИССЛЕДОВАТЕЛЬСКОЙ РАБОТЕ}\\
~\\
SYNCHRONIZATION OF NEUROMORPHIC NETWORKS OF THE CLOSE WORLD FROM THE POINT OF VIEW OF COMPLEXES\\
(заключительный)\\
\vspace*{\fill}
\end{center}
\zz{~}Выполнил:
\zz{~}Студент группы БПИ204
\zz{~}образовательной программы
\zz{~}«Программная инженерия»
\zz{~}Пеганов Никита Сергеевич
\zz{~}\noindent\rule{3cm}{0.4pt} Н. С. Пеганов
\zz{~}«\noindent\rule{1cm}{0.4pt}»\noindent\rule{2cm}{0.4pt}20\noindent\rule{0.5cm}{0.4pt}г.
\begin{center}
\vspace*{\fill}{
  Москва 2022}
\end{center}
\newpage
\begin{center}
\section {Реферат}
\end{center}

\newpage
\begin{center}
\section {Содержание}
\tableofcontents
\end{center}
\newpage
\section {Основные термины, определения и сокращения}



\newpage
\begin{center}
\item\section {Введение}
\end{center}
\textbf{Описание задачи}\\
~\\
\textbf{Актуальность}\\
~\\
\textbf{Предмет исследования}\\
~\\
\textbf{Методы исследования}\\
~\\
\textbf{Цели и задачи работы}\\
~\\
\textbf{Новизна и достоверность полученных результатов}\\
~\\
\textbf{Теоретическая значимость}\\
~\\
\textbf{Практическая ценность}\\
~\\
\newpage
\begin{center}
\section {Основная часть отчёта о НИР}
\end{center}
\textbf{Обзор и анализ источников}\\
~\\
\textbf{Выбор методов, алгоритмов, моделей для решения поставленных задач}\\
~\\
\textbf{Описание выбранных или предлагаемых методов, алгоритмов, моделей, методик}\\
~\\
\textbf{Описание эксперимента}\\
~\\
\textbf{Обзор и анализ источников}\\
~\\
\textbf{Описание эксперимента}\\
~\\

\newpage
\begin{center}
\section {Заключение}
\end{center}

\newpage
\begin{center}
\begin{thebibliography}{}
\bibitem{litlink24} \textit{Ning Liu, Zhe Li, Zhiyuan Xu, Jielong Xu, Sheng Lin, Qinru Qiu, Jian Tang, Yanzhi Wang} (2017) A Hierarchical Framework of Cloud Resource Allocation and Power Management Using Deep Reinforcement Learning. // Сайт Arxiv.org. 13 марта (https://arxiv.org/abs/1703.04221) Просмотрено: 05.05.2022
\end{thebibliography}
\end{center}
\newpage
\begin{center}
\section {Приложения}
\end{center}
\zz{}\textbf{Приложение 1\\}
Ссылка на репозиторий проекта с исходным кодом и всеми использованными материалами.\\
https://github.com/NikPeg/synchronization-of-neuromorphic-networks-of-the-close-world-from-the-point-of-view-of-complexes\\
\end{document}